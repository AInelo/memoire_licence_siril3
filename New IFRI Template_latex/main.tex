%--------------A VOTRE ATTENTION-------------%
% Les étudiants en master qui disposent de plus de 3 chapitres dans leurs travaux peuvent en complèter
% Les Membres doivent figurer dans la dernière version finale du mémoire après soutenance pour dépôt de mémoire

\documentclass{ifri}
\usepackage{titletoc}
\setlength{\glsdescwidth}{0.65\textwidth}
% \usepackage{lscape}

\typeMemoire{Diplôme de Licence en Informatique}
\optionFormation{Système d’Informations et Réseaux Informatiques}
\etudiant{Lionel \textbf{TOTON}}
\titreDuMemoire{Mise en place d'une application web génératrice de plan d'affaires adapté au monde l'entreprenariat Béninois} %Implémention pour une meilleure sécurité dans les réseau LAN sous IPv6 // Proposition: Identification des vulnérabilités dans un reseau LAN IPv6 et mesures pour une meilleure sécurité.

\dateSoutenance{-}
%\promo{2\up{ème}}
\anneeScolaire{--}


%%maitre de mémoire
\encadrants{Prenom \textbf{Nom}}

%% Membres du Jury
\jurys{%
\begin{tabular}{llll}
	Nom et prénoms du président & Grade & Entité & Président \\
	Nom et prénoms de l'examinateur & Grade & Entité & Examinateur \\
	Nom et prénoms du rapporteur & Grade & Entité & Rapporteur \\
\end{tabular}	
}


\hypersetup{
 pdftitle={--},
 pdfauthor={--},
 pdfsubject={--},
 pdfkeywords={--} }


\color{bookColor}

%importation du glossaire
\loadglsentries{glossaire_reduit}

\begin{document}

\pageDeGarde
%\pageTitre

\pagecolor{white}

%% page vide
%\thispagestyle{empty}\ \clearpage


\selectlanguage{french}

% sommaire
\pagenumbering{roman}

\setcounter{tocdepth}{0}
\startlist{toc}
\printlist{toc}{}{\chapter*{Sommaire}}
\setcounter{tocdepth}{5}

%% rdedicaces
\dedicace
Mes dedicaces à 



\newpage 

%% remerciements
\remerciements

Nos remerciements
\newpage 

% Résume
\resume
\selectlanguage{french}
\vspace*{-6cm}
\begin{abstract}
    L'entrepreneuriat est en pleine expansion au Bénin, stimulant une croissance économique soutenue. Cependant, de nombreux entrepreneurs béninois se heurtent au défi de la création de plans d'affaires solides et adaptés aux spécificités du marché local. Ce mémoire se penche sur la conception et le développement d'une application web novatrice, spécifiquement dédiée à la génération de plans d'affaires adaptés à la réalité béninoise.

    Le mémoire commence par une analyse approfondie du contexte entrepreneurial au Bénin, explorant les secteurs clés, les opportunités et les obstacles auxquels font face les entrepreneurs locaux. Cette analyse sert de fondement à la création d'une application web conviviale et accessible, conçue pour simplifier et démocratiser le processus de planification d'entreprise.
    
    L'application intègre des fonctionnalités essentielles, telles que l'évaluation des risques, la planification financière, la recherche de marché et la personnalisation des plans d'affaires en fonction des besoins spécifiques des utilisateurs. Elle vise à améliorer l'efficacité des entrepreneurs dans leur processus de planification, contribuant ainsi à la viabilité et à la pérennité de leurs entreprises.
    
    Ce mémoire explore également les enjeux techniques, de sécurité des données, financiers et éthiques liés au développement de l'application. Il s'efforce de répondre aux questions essentielles concernant la protection des données des utilisateurs, la durabilité financière du projet et l'impact positif qu'il peut avoir sur l'entrepreneuriat béninois.
    
    En fin de compte, ce projet aspire à démocratiser l'accès à des plans d'affaires de qualité, à stimuler l'entrepreneuriat local, à favoriser la croissance économique et à contribuer à la prospérité du Bénin. Il ouvre également la voie à d'autres initiatives similaires dans la région, soulignant l'importance de l'innovation technologique dans le soutien à l'entrepreneuriat en Afrique de l'Ouest.
    
    

\paragraph{}
\textbf{Mots clés}: ....,....,....
\end{abstract}

\newpage
\thispagestyle{empty}
\selectlanguage{english}
\addcontentsline{toc}{chapter}{Abstract}
\begin{abstract}

    Entrepreneurship is booming in Benin, stimulating sustained economic growth. However, many Beninese entrepreneurs face the challenge of creating solid business plans adapted to the specificities of the local market. This dissertation focuses on the design and development of an innovative web application, specifically dedicated to the generation of business plans adapted to the Beninese reality.

    The dissertation begins with an in-depth analysis of the entrepreneurial context in Benin, exploring the key sectors, opportunities and obstacles faced by local entrepreneurs. This analysis serves as the foundation for creating a user-friendly and accessible web application designed to simplify and democratize the business planning process.
    
    The application integrates essential features, such as risk assessment, financial planning, market research and customization of business plans according to specific user needs. It aims to improve the efficiency of entrepreneurs in their planning process, thus contributing to the viability and sustainability of their businesses.
    
    This dissertation also explores the technical, data security, financial and ethical issues related to the development of the application. It strives to answer essential questions regarding the protection of user data, the financial sustainability of the project and the positive impact it can have on Beninese entrepreneurship.
    
    Ultimately, this project aspires to democratize access to quality business plans, stimulate local entrepreneurship, promote economic growth and contribute to the prosperity of Benin. It also paves the way for other similar initiatives in the region, highlighting the importance of technological innovation in supporting entrepreneurship in West Africa.

\paragraph{}
\textbf{Key words}: ....,....,....
\end{abstract}
\newpage

%liste des figures
\listoffigures 
\newpage

%liste des tableaux
\listoftables
\newpage

%liste des algo
\selectlanguage{french}
\listofalgorithmes
\newpage

% Les sigles et acronymes
\setglossarystyle{altlist}
\printglossary[title=Liste des acronymes, toctitle=Liste des acronymes, type=\acronymtype]
\newpage

% Le glossaire proprement dit
%\setglossarystyle{super}
%\printglossary[type=main]


\pagenumbering{arabic}
\setcounter{page}{1}
%%introduction
\introduction
Le Bénin, comme de nombreux pays en développement, est le théâtre d'une croissance économique continue, stimulée par l'entrepreneuriat et l'innovation. Alors que de plus en plus de Béninois aspirent à devenir des entrepreneurs prospères, la nécessité d'un plan d'affaires solide et adapté à la réalité locale devient primordiale. Un plan d'affaires bien conçu est le socle sur lequel repose toute entreprise prospère, et il joue un rôle crucial dans la navigation des défis et opportunités du marché béninois.

Cependant, la création d'un plan d'affaires efficace est souvent perçue comme une tâche complexe et intimidante, en particulier pour les entrepreneurs novices qui manquent d'expérience en rédaction de plans d'affaires ou de ressources pour engager des consultants coûteux. C'est dans ce contexte qu'émerge la nécessité de développer une solution innovante qui simplifie et démocratise le processus de création de plans d'affaires, en mettant à la disposition des entrepreneurs une application web conviviale et accessible.

Ce mémoire vise à explorer en détail la conception, le développement et l'implémentation d'une application web dédiée à la génération de plans d'affaires, spécialement adaptée aux besoins et aux réalités du Bénin. Nous plongerons dans l'analyse des spécificités du marché béninois, des secteurs clés et des défis auxquels sont confrontés les entrepreneurs locaux. À partir de cette base, nous élaborerons un modèle conceptuel pour l'application, en intégrant des fonctionnalités essentielles telles que l'évaluation des risques, la planification financière, la recherche de marché, et bien plus encore.

Notre objectif est de démocratiser l'accès à des plans d'affaires de qualité en fournissant aux entrepreneurs béninois un outil puissant, intuitif et personnalisable qui les guide dans la création de plans d'affaires qui répondent aux besoins spécifiques de leurs entreprises. En permettant aux entrepreneurs de gagner en efficacité dans le processus de planification, notre application ambitionne de stimuler l'entrepreneuriat, de favoriser la croissance économique et de contribuer à la prospérité du Bénin.

Ce mémoire explorera également les défis techniques, les considérations de sécurité des données, les aspects financiers et les implications éthiques liées au développement de cette application web. En fin de compte, nous espérons que ce projet offrira une contribution significative au domaine de l'entrepreneuriat au Bénin et servira de modèle pour d'autres initiatives similaires dans la région.  \gls{acro} puis \Gls{acroglo} et enfin \gls{glossaire}
%\lhead[]{} \rhead[]{} \chead[]{}
\selectlanguage{french}
\fancyhead[L]{\tiny \leftmark}
\fancyhead[R]{\scriptsize \rightmark}
\fancyfoot[C]{\thepage}

\chapter{-}\label{chap:1}
 \addcontentsline{toc}{section}{Introduction}
\section*{Introduction}

\section{-}
blablabla

\section{-}
blablabla

\addcontentsline{toc}{section}{Conclusion}
\section*{Conclusion}

 
 \chapter{-}\label{chap:2}
 \addcontentsline{toc}{section}{Introduction}
\section*{Introduction}

\section{-}
blablabla

\section{-}
blablabla

\addcontentsline{toc}{section}{Conclusion}
\section*{Conclusion}

 
\chapter{--}\label{chap:3}
 \addcontentsline{toc}{section}{Introduction}
\section*{Introduction}

\section{-}
blablabla

\section{-}
blablabla

\addcontentsline{toc}{section}{Conclusion}
\section*{Conclusion}

 
% \include{perspectives}
%%conclusion
\conclusion
Bla bla bla \cite{ehrig2006graph}
% 
\lhead[]{} \rhead[]{} \chead[]{}

%%biblio
\addcontentsline{toc}{chapter}{Bibliographie}
\bibliographystyle{abbrv}
\bibliography{biblio}


%\chapter*{Annexe}\addcontentsline{toc}{chapter}{Annexe}\label{annexe1}

\subsection*{Étapes clés du déroulement de l'attaque}


Nous allons exploiter quelques failles de ce réseau pour effectuer une attaque man in the middle (MITM).\\

Au début, notre machine Windows peut atteindre normalement le routeur R4.
\begin{figure}[H]
    \centering
    \includegraphics[scale=0.8]{images/ping_b4_1}
    \caption{Ping vers le routeur R4 avec succès}
    \label{fig:ping_b4_1}
\end{figure}
Quand on essaie de tracer le chemin vers R4, on constate que la machine passe par le routeur R1 légitime du lien pour atteindre R4
\begin{figure}[H]
    \centering
    \includegraphics{images/tracert_b4_1}
    \caption{Traces du chemin vers R4}
    \label{fig:tracert_b41}
\end{figure}
L'attaquant sur le lien peut alors passer a l'attaque.
Pour effectuer l'attaque MITM on utilisera l'outil fake\_router6, un utilitaire du package d'outils \textbf{the hacker choice}.
Ainsi sur la machine d'attaque, on active en un premier lieu le forwarding pour être transparent et ne pas bloquer le transit des paquets.
\begin{figure}[H]
    \centering
    \includegraphics{images/attk/fwrd_activation}
    \caption{Activation du forwarding des paquets.}
    \label{fig:activ_fwrd}
\end{figure}
Aussi on lance wireshark pour observer le trafic des paquets sur notre interface dans le réseau.\\
-------\\
Puisque tout est prêt nous allons lancer l'attaque.

\begin{figure}[H]
    \centering
    \includegraphics[scale=0.8]{images/attk/lancement_attk_1}
    \caption{Initialisation de l'attaque}
    \label{fig:attk_init_1}
\end{figure}

L'attaque est en cours et l'attaquant s'annonce comme le routeur par défaut du lien
nous allons maintenant vérifier la table des routes de notre machine windows.
\begin{figure}[H]
    \centering
    \includegraphics{images/attk/tableRoutes_windows}
    \caption{Table des routes de la machine victime}
    \label{fig:win_route_table}
\end{figure}
On constate que l'attaquant s'est insère comme passerelle de la victime.
pour confirmer cela reprenons un tracert vers le routeur r4
\begin{figure}[H]
    \centering
    \includegraphics{images/attk/tracert_b4_2}
    \caption{Chemin vers b4 pendant l'attaque.}
    \label{fig:tracert_b42}
\end{figure}
On peut voir clairement que la victime passe par l'attaquant pour atteindre le routeur.\\

A présent nous allons essayer de capturer une information envoyée par la victime.
Pour cela la victime fait un telnet sur le router R4 pour s'y connecter avec les paramètres suivants:\\
password1:\textbf{cisco}\\
password2:\textbf{class}
\begin{figure}[H]
    \centering
    \includegraphics{images/attk/telnet_r4}
    \caption{Connexion telnet au routeur.}
    \label{fig:telnetr4}
\end{figure}

Une fois la connexion réussie, nous allons voir avec wireshark les paquets de connexion et y retrouver les paramètres de connexion.
\begin{figure}[H]
    \centering
    \includegraphics[width=1.0\textwidth]{images/attk/c}
    \includegraphics[width=1.0\textwidth]{images/attk/i}
    \includegraphics[width=1.0\textwidth]{images/attk/s}
    \includegraphics[width=1.0\textwidth]{images/attk/c2}
    \includegraphics[width=1.0\textwidth]{images/attk/o}   
    \caption{Premier paramètre de connexion au routeur R4: \textbf{c-i-s-c-o}}
    \label{fig:param_conn_r4}
\end{figure}
\begin{figure}[H]
    \centering
    \includegraphics[width=1.0\textwidth]{images/attk/param2_c}
    \includegraphics[width=1.0\textwidth]{images/attk/param2_l}
    \includegraphics[width=1.0\textwidth]{images/attk/param2_a}
    \includegraphics[width=1.0\textwidth]{images/attk/param2_s1}
    \includegraphics[width=1.0\textwidth]{images/attk/param2_s2}
    \caption{Second paramètre de connexion au routeur R4: \textbf{c-l-a-s-s}}
    \label{fig:param_conn2}
\end{figure}
Les paramètres on été retrouves donc l'attaque a été un succès!

%\subsection*{Mitigations}
%Pour sécuriser ce réseau afin d'éviter ce genre d'attaque, deux mesures de sécurité peuvent être configurées.
%\begin{itemize}
%    \item le SEND
%    \item le RaGuard
%\end{itemize}



\newpage
\tableofcontents

\end{document}          
