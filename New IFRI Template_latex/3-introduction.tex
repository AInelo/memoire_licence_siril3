\introduction
Le Bénin, comme de nombreux pays en développement, est le théâtre d'une croissance économique continue, stimulée par l'entrepreneuriat et l'innovation. Alors que de plus en plus de Béninois aspirent à devenir des entrepreneurs prospères, la nécessité d'un plan d'affaires solide et adapté à la réalité locale devient primordiale. Un plan d'affaires bien conçu est le socle sur lequel repose toute entreprise prospère, et il joue un rôle crucial dans la navigation des défis et opportunités du marché béninois.

Cependant, la création d'un plan d'affaires efficace est souvent perçue comme une tâche complexe et intimidante, en particulier pour les entrepreneurs novices qui manquent d'expérience en rédaction de plans d'affaires ou de ressources pour engager des consultants coûteux. C'est dans ce contexte qu'émerge la nécessité de développer une solution innovante qui simplifie et démocratise le processus de création de plans d'affaires, en mettant à la disposition des entrepreneurs une application web conviviale et accessible.

Ce mémoire vise à explorer en détail la conception, le développement et l'implémentation d'une application web dédiée à la génération de plans d'affaires, spécialement adaptée aux besoins et aux réalités du Bénin. Nous plongerons dans l'analyse des spécificités du marché béninois, des secteurs clés et des défis auxquels sont confrontés les entrepreneurs locaux. À partir de cette base, nous élaborerons un modèle conceptuel pour l'application, en intégrant des fonctionnalités essentielles telles que l'évaluation des risques, la planification financière, la recherche de marché, et bien plus encore.

Notre objectif est de démocratiser l'accès à des plans d'affaires de qualité en fournissant aux entrepreneurs béninois un outil puissant, intuitif et personnalisable qui les guide dans la création de plans d'affaires qui répondent aux besoins spécifiques de leurs entreprises. En permettant aux entrepreneurs de gagner en efficacité dans le processus de planification, notre application ambitionne de stimuler l'entrepreneuriat, de favoriser la croissance économique et de contribuer à la prospérité du Bénin.

Ce mémoire explorera également les défis techniques, les considérations de sécurité des données, les aspects financiers et les implications éthiques liées au développement de cette application web. En fin de compte, nous espérons que ce projet offrira une contribution significative au domaine de l'entrepreneuriat au Bénin et servira de modèle pour d'autres initiatives similaires dans la région.  \gls{acro} puis \Gls{acroglo} et enfin \gls{glossaire}