\resume
\selectlanguage{french}
\vspace*{-6cm}
\begin{abstract}
    L'entrepreneuriat est en pleine expansion au Bénin, stimulant une croissance économique soutenue. Cependant, de nombreux entrepreneurs béninois se heurtent au défi de la création de plans d'affaires solides et adaptés aux spécificités du marché local. Ce mémoire se penche sur la conception et le développement d'une application web novatrice, spécifiquement dédiée à la génération de plans d'affaires adaptés à la réalité béninoise.

    Le mémoire commence par une analyse approfondie du contexte entrepreneurial au Bénin, explorant les secteurs clés, les opportunités et les obstacles auxquels font face les entrepreneurs locaux. Cette analyse sert de fondement à la création d'une application web conviviale et accessible, conçue pour simplifier et démocratiser le processus de planification d'entreprise.
    
    L'application intègre des fonctionnalités essentielles, telles que l'évaluation des risques, la planification financière, la recherche de marché et la personnalisation des plans d'affaires en fonction des besoins spécifiques des utilisateurs. Elle vise à améliorer l'efficacité des entrepreneurs dans leur processus de planification, contribuant ainsi à la viabilité et à la pérennité de leurs entreprises.
    
    Ce mémoire explore également les enjeux techniques, de sécurité des données, financiers et éthiques liés au développement de l'application. Il s'efforce de répondre aux questions essentielles concernant la protection des données des utilisateurs, la durabilité financière du projet et l'impact positif qu'il peut avoir sur l'entrepreneuriat béninois.
    
    En fin de compte, ce projet aspire à démocratiser l'accès à des plans d'affaires de qualité, à stimuler l'entrepreneuriat local, à favoriser la croissance économique et à contribuer à la prospérité du Bénin. Il ouvre également la voie à d'autres initiatives similaires dans la région, soulignant l'importance de l'innovation technologique dans le soutien à l'entrepreneuriat en Afrique de l'Ouest.
    
    

\paragraph{}
\textbf{Mots clés}: ....,....,....
\end{abstract}

\newpage
\thispagestyle{empty}
\selectlanguage{english}
\addcontentsline{toc}{chapter}{Abstract}
\begin{abstract}

    Entrepreneurship is booming in Benin, stimulating sustained economic growth. However, many Beninese entrepreneurs face the challenge of creating solid business plans adapted to the specificities of the local market. This dissertation focuses on the design and development of an innovative web application, specifically dedicated to the generation of business plans adapted to the Beninese reality.

    The dissertation begins with an in-depth analysis of the entrepreneurial context in Benin, exploring the key sectors, opportunities and obstacles faced by local entrepreneurs. This analysis serves as the foundation for creating a user-friendly and accessible web application designed to simplify and democratize the business planning process.
    
    The application integrates essential features, such as risk assessment, financial planning, market research and customization of business plans according to specific user needs. It aims to improve the efficiency of entrepreneurs in their planning process, thus contributing to the viability and sustainability of their businesses.
    
    This dissertation also explores the technical, data security, financial and ethical issues related to the development of the application. It strives to answer essential questions regarding the protection of user data, the financial sustainability of the project and the positive impact it can have on Beninese entrepreneurship.
    
    Ultimately, this project aspires to democratize access to quality business plans, stimulate local entrepreneurship, promote economic growth and contribute to the prosperity of Benin. It also paves the way for other similar initiatives in the region, highlighting the importance of technological innovation in supporting entrepreneurship in West Africa.

\paragraph{}
\textbf{Key words}: ....,....,....
\end{abstract}